\documentclass[12pt]{book}
\usepackage[utf8]{inputenc}


\usepackage{amsfonts}
\usepackage{amsmath}
\usepackage{amssymb}
\usepackage{amsthm}
\usepackage{float}
\usepackage{graphicx}
\usepackage{listings}
\usepackage{multirow}
\usepackage[T1]{fontenc}
\usepackage[spanish]{babel}
\usepackage[table,xcdraw]{xcolor}
\usepackage[pages=some,scale=1,angle=0,opacity=0.2]{background}
\newcommand\BackImage[2][scale=1]{%
\BgThispage
\backgroundsetup{
  contents={\includegraphics[#1]{#2}}
  }
}
\newcommand{\C}{\mathbb{C}}
\newcommand{\R}{\mathbb{R}}
\newenvironment{solution}
  {\renewcommand\qedsymbol{$\square$}\begin{proof}[Solución]}
  {\end{proof}}
\newtheorem{exercise}{Ejercicio}

\begin{document}
\thispagestyle{empty}
\vfill
\begin{center}
\hrule
\vspace{0.2cm}
\large{\textsc{--- Universidad de San Carlos de Guatemala ---}}\\
\large{\textsc{Escuela de Ciencias Físicas y Matemáticas}}\\
\large{\textsc{Departamento de matemática}}\\
\begin{large}
\textsc{\textbf{Análisis Numérico 1}}\\
\end{large}
Profesor: \textit{Damian Ochoa}\\
\vspace{0.2cm}
\hrule
\end{center}
\BackImage[width=0.8\textwidth]{img}
\vfill
\begin{center}
\Large{\textsc{\textbf{Tarea extra}}}\\
%\Huge{\textsc{\textbf{IOT}}}\\
\end{center}
\vfill
\begin{center}
\hrule
\vspace{0.07cm}
\hrule
\vspace{0.2cm}
\begin{flushright}
\begin{large}
\textit{\textbf{Javier Estuardo Navarro Delgado}}\\
\textit{\textbf{2015-13630}}\\
\textit{\textbf{13 de abril de 2020}}\\
\end{large}
\end{flushright}
\vspace{0.2cm}
\hrule
\vspace{0.07cm}
\hrule
\vspace{1cm}
\end{center}
\tableofcontents
\chapter{Repaso: Algo de Álgebra Lineal}
\section{Vectores en $\R^n$ y $\C^n$}
\exercise Indicar las propiedades de un producto interno/producto escalar para un espacio vectorial complejo V, y verificar que el producto interno euclidiano para $\C^n$ tiene estas propiedades.
\exercise Demostrar que cualquier vector $x\in\C^n$ tiene la representación
\[ x = \sum^n_{j=1}(\textbf{v}^*_jx)\textbf{v}_j\]
como una combinación lineal con respecto a la base ortonormal $\textbf{v}_1, \textbf{v}_2, \dots, \textbf{v}_n$.

\section{Matrices}
\exercise Mostrar que $\C^{m\times n}$ con la habitual suma de matrices y la multiplicación escalar es un espacio vectorial complejo.

\exercise Encontrar el rango y el espacio nulo de la siguiente matriz:
\[
A:=\begin{pmatrix}
1 & 0 & -1 & 2\\
0 & 1 & 3 & 1\\
-1 & 1 & 5 & 0
\end{pmatrix}
\]
\begin{solution} Para obtener el rango triangularizaremos la matriz y los vectores independientes de la nueva matriz serán el rango:
\begin{align*}
    \begin{pmatrix}
    1 & 0 & -1 & 2\\
0 & 1 & 3 & 1\\
-1 & 1 & 5 & 0
\end{pmatrix} & \rightarrow & \begin{pmatrix}
1 & 0 & -1 & 2\\
0 & 1 & 3 & 1\\
0 & 1 & 4 & 0
\end{pmatrix} & \rightarrow & \begin{pmatrix}
1 & 0 & -1 & 2\\
0 & 1 & 3 & 1\\
0 & 0 & 1 & -1
\end{pmatrix}
\end{align*}

Es fácil notar que la cuarta columna de la matriz triangular es combinación lineal de las primeras tres columnas. Así que:
\[
range(A)=span\left\{\begin{pmatrix}
1\\
0\\
0
\end{pmatrix}, \begin{pmatrix}
0\\
1\\
0
\end{pmatrix}, \begin{pmatrix}
-1\\
3\\
1
\end{pmatrix}\right\}.
\]

Para obtener el kernell de $A$:
\begin{gather*}
    \begin{pmatrix}
    1 & 0 & -1 & 2\\
0 & 1 & 3 & 1\\
-1 & 1 & 5 & 0
\end{pmatrix} \begin{pmatrix}
w\\
x\\
y\\
z
\end{pmatrix} = \begin{pmatrix}
0\\
0\\
0\\
0
\end{pmatrix}\\
\Rightarrow \begin{cases}
w-y+2z=0\\
x+3y+z=0\\
-w+x+5y=0
\end{cases} \Rightarrow \begin{cases}
w=-3z\\
x=-2y\\
y=-z
\end{cases}
\end{gather*}

Si elegimos $z=1$:
\[
ker(A)=span\left\{\begin{pmatrix}
-3\\
2\\
-1\\
1
\end{pmatrix}
\right\}.
\]
\end{solution}

\exercise Para una matriz $A\in\C^{m\times n}$ mostrar que el rango de $A$ es el espacio lineal abarcado por los vectores de la columna de $A$.
\begin{solution}
Sea $A\in\C^{m\times n}$. Sabemos que el rango de $A$ es el conjunto de todos los vectores $Ax$, donde $x\in\C^n$. Así que:
\[
Ax=\begin{pmatrix}
a_{1, 1} & \dots & a_{1, n}\\
\vdots & \ddots & \vdots\\
a_{m, 1} & \dots & a_{m, n}
\end{pmatrix}\begin{pmatrix}
x_1\\
\vdots\\
x_n
\end{pmatrix}=\begin{pmatrix}
a_{1,1}x_1+a_{1,2}x_2+\dots+a_{1,n}x_n\\
\vdots\\
a_{m,1}x_1+a_{m,2}x_2+\dots+a_{m,n}x_n
\end{pmatrix}
\]
Definanos $a_i$ como la i-ésima columna de A:
\[
a_i=\begin{pmatrix}
a_{i, i}\\
\vdots\\
a_{m, i}
\end{pmatrix}
\]
Entonces:
\begin{align*}
    Ax&=\begin{pmatrix}
a_{1,1}x_1+a_{1,2}x_2+\dots+a_{1,n}x_n\\
\vdots\\
a_{m,1}x_1+a_{m,2}x_2+\dots+a_{m,n}x_n
\end{pmatrix}\\
&= x_1\begin{pmatrix}
a_{1, 1}\\
\vdots\\
a_{m, 1}
\end{pmatrix} + x_2\begin{pmatrix}
a_{1, 2}\\
\vdots\\
a_{m, 2}
\end{pmatrix} + \dots + x_n\begin{pmatrix}
a_{1, n}\\
\vdots\\
a_{m, n}
\end{pmatrix}\\
&=x_1a_1+x_2a_2+\dots+x_na_n
\end{align*}
Es decir que cada vector $Ax$ se puede representar como una combinación lineal de las columnas de A:
\[
range(A)=span\{a_i\}
\]
\end{solution}

\exercise ¿Cuál de las siguientes matrices cuadradas, si es que hay alguna, es simétrica o Hermitiana?
\begin{align*}
    A&:=\begin{pmatrix} 
    -1 & 2 & i\\
    2 & i & 3\\
    -i & 3 & 1
    \end{pmatrix} & B&:=\begin{pmatrix}
    -1 & 2 & -i\\
    2 & 7 & 5\\
    i & 5 & 3
    \end{pmatrix} & C&:=\begin{pmatrix}
    2 & -2 & 8\\
    -2 & 7 & -1\\
    8 & -1 & 3
    \end{pmatrix}
\end{align*}

\begin{solution}
\begin{gather*}
    A^*=\begin{pmatrix}
    -1 & 2 & i\\
    2 & -i & 3\\
    i & 3 & 1
    \end{pmatrix}\neq A\\
    B^*=\begin{pmatrix}
    -1 & 2 & -i\\
    2 & 7 & 5\\
    i & 5 & 3
    \end{pmatrix}=\Rightarrow Hermitiana\\
    C^*=C\Rightarrow Sim\acute{e}trica
\end{gather*}
\end{solution}

\exercise Mostrar que $(AB)^T = B^TA^T$ para cualquier $A\in\R^{m\times n}$ y $B\in\R^{n\times p}$. Mostrar que $(AB)^*= B^{*} A^{*}$ para cualquier $A\in\C^{m\times n}$ y $B\in\C^{n\times p}$.
\begin{solution}
Sean $A=(a_{i,j})\in\R^{m\times n}$ y $B=(b_{i,j})\in\R^{n\times p}$, $(AB)^T=[(a_{i,j})(b_{i,j})]^T=(c_{i,j})^T$, donde $c_{i,j}=\sum_{k=1}^na_{i,k}b_{k,j}$ ($i=1, \dots, m; j=a, \dots, p$). Entonces,
\begin{align*}
    (AB)^T&=(c_{i,j})=\left(\sum_{k=1}^na_{i,k}b_{k,j}\right)^T=\left(\sum_{k=1}^nb_{k,j}a_{i,k}\right)\\
    &=(b_{j,i})(a_{j,i})=(b_{i,j})^T(a_{i,j})^T=B^TA^T
\end{align*}
También,
\begin{align*}
    (AB)^*&=\overline{(AB)^T}=\overline{B^TA^T}=\left(\overline{\sum_{k=1}^nb_{k,j}a_{i,k}}\right)\\
    &=\left(\sum_{k=1}^n\overline{b_{k,j}a_{i,k}}\right)=\left(\sum_{k=1}^n\overline{b_{k,j}} \overline{a_{i,k}}\right)=\overline{B}^T \overline{A}^T\\
    &=B^*A^*
\end{align*}
\end{solution}

\exercise Calcular la traza de la matriz de $3\times3$:
\[
A=\begin{pmatrix}
\frac{3}{2} & 0 & \frac{1}{2}\\
0 & 3 & 0\\
\frac{1}{2} & 0 & \frac{3}{2}
\end{pmatrix}.
\]
\begin{solution}
Para esto calculemos los valores propios ($\lambda_i$, $i=1, 2, 3$) de $A$:
\begin{align*}
    det(A-\lambda I)&=0\\
    \begin{vmatrix}
\frac{3}{2}-\lambda & 0 & \frac{1}{2}\\
0 & 3-\lambda & 0\\
\frac{1}{2} & 0 & \frac{3}{2}-\lambda
\end{vmatrix} &=0\\
\left(\frac{3}{2}-\lambda\right)^2(3-\lambda)-\frac{1}{4}(3-\lambda)&=0\\
(3-\lambda)\left[\left(\frac{3}{2}-\lambda\right)^2-\frac{1}{4}\right]&=0\\
\end{align*}
Así que:
\begin{align*}
    3-\lambda&=0 & \left(\frac{3}{2}-\lambda\right)^2-\frac{1}{4}&=0\\
    \lambda_1&=3 & \lambda&=\frac{3}{2}\pm\frac{1}{2}\\
    & & \lambda_2&=2\\
    & & \lambda_3&=1
\end{align*}
Finalmente, $traza(A)=\lambda_1+\lambda_2+\lambda_3=3+2+1=6$.
\end{solution}

\exercise Demostrar que la matriz A de $3\times3$ del ejercicio anterior es definida positiva.
\begin{solution}
Tomemos $x\in\R$, $x\neq0$:
\begin{align*}
    x&=\begin{pmatrix}
    x_1\\
    x_2\\
    x_3
    \end{pmatrix} & \Rightarrow x^TAx&=\begin{pmatrix}
    x_1 & x_2 & x_3
    \end{pmatrix}\begin{pmatrix}
    \frac{3}{2} & 0 & \frac{1}{2}\\
0 & 3 & 0\\
\frac{1}{2} & 0 & \frac{3}{2}
\end{pmatrix}\begin{pmatrix}
    x_1\\
    x_2\\
    x_3
    \end{pmatrix}\\
    & & &=\frac{3}{2}x^2+\frac{1}{2}xz+3y^2+\frac{1}{2}xz+\frac{3}{2}z^2\\
    & & &=\frac{3}{2}x^2+\frac{3}{2}z^2+xz+3y^2\\
    & & &=\frac{1}{2}[(x+y)^2]+2x^2+2z^2+6y^2]
\end{align*}
Sabemos que $(x+y)^2\geq0$, $x^2\geq0$, $y^2\geq0$ y $z^2\geq0$. Así que $x^TAx>0$, y $A$ es definida positiva.
\end{solution}


\exercise Sea $A\in\C^{n\times n}$ una matriz definida positiva. Si $C\in\C^{n\times m}$, demostrar que:
\renewcommand{\labelenumi}{(\alph{enumi})}
\begin{enumerate}
    \item $C^*AC$ es definida semi-positiva.
    \item $rank(C^*AC)= rank(C)$.
    \item $C^*AC$ es definida positiva ssi $rank(C)=m$.
\end{enumerate}
\begin{solution}
\renewcommand{\labelenumi}{(\alph{enumi})}
\begin{enumerate}
    \item Sea $x\in\C^m$, $x\neq0$. $x^*C^*ACx=y^*Ay$, donde $y\in\C^n$, $y\geq0$. Así que $C^*AC$ es semidefinida positiva.
    \item Sea $x\in\C^m$, $C^*ACx=C^*AY$, donde $y\in\C^n$. Entonces, $C^*Ay=C^*z$, donde $z\in\C^n$. Luego,
    \begin{align*}
        rank(C^*AC)&=dim\left(range(C^*AC)\right)=dim\left(\{y:C^*AC=y\}\right)\\
        &=dim\left(\{y:C^*x=y\}\right)=dim(rangeC^*)
    \end{align*}
    Notemos que
    \begin{align*}
        dim(rangeC)&=dim(rangeC^T)=dim(range\Bar{C^T}\\
        &=dim(rangeC^*)
    \end{align*}
    \item
\end{enumerate}
\end{solution}
\section{Determinantes de matrices cuadradas}
\exercise Calcule el determinante de la matriz de $3\times3$:
\[
A=\begin{pmatrix}
\frac{3}{2} & 0 & \frac{1}{2}\\
0 & 3 & 0\\
\frac{1}{2} & 0 & \frac{3}{2}
\end{pmatrix}.
\]
\begin{solution}
$det(A) = ( \frac{3}{2}\times3\times\frac{3}{2})-(\frac{1}{2}\times3\times\frac{1}{2})= 6$
\end{solution}
\exercise Demostrar que si $A=(a_{i,j})\in \R^{n\times n}$ es definida positiva, entonces las submatrices principales superiores $A_p := (a_{i,j})_{1\leq i, j\leq p}$, $p\in\{1,2,...,n\}$, son definidas positivas.
\exercise Demostrar que para todas la matrices cuadradas A de n$\times$n, $det(A)=det(A^T)$.
\begin{solution}
Consideremos dos casos: $detA=0$ y $detA\neq0$.\\
Si $detA=0$, $A$ no es invertible. Esto implica que $A^T$ tampoco lo es. Así que $detA^T=0$. Por lo tanto, $detA^T=detA$.\\
Ahora, si $detA\neq0$, $A$ es invertible, por lo que puede ser representada como el producto de matrices elementales: $E_1E_2\cdots E_k$. 
\end{solution}
\section{Matríz inversa de una matriz cuadrada}
\exercise Mostrar que la matriz de 3 $\times$ 3: 
\[
A=\begin{pmatrix}
\frac{3}{2} & 0 & \frac{1}{2}\\
0 & 3 & 0\\
\frac{1}{2} & 0 & \frac{3}{2}
\end{pmatrix},
\]
es no singular. Calcule la matriz inversa $A^{-1}$ de A.

\exercise Sea $A, B\in\C^{n\times n}$ matrices invertibles. Demostrar lo siguiente:
\renewcommand{\labelenumi}{(\alph{enumi})}
\begin{enumerate}
    \item $(AB)^{-1}=(A)^{-1}(B)^{-1}$
    \item $rank(C^*AC)= rank(C)$.
    \item $C^*AC$ es definida positiva ssi $rank(C)=m$.
\end{enumerate}



\end{document}
